\chapter{Future work}
\label{chap:future-work}
At the moment the \gls{poc} includes an entire Kubernetes cluster, so that's one obvious point of improvement towards a final solution. As future work we could implement our own api-server for the network objects and make it run highly available on all the masters in a \gls{dcos} cluster. Furthermore we need to extend the dcos-cli to include the new network objects that we have defined. Once this is in place we can extend the web interface of \gls{dcos} to query the api-server for network objects and display them in the web interface. This will help developers and operators to check which virtual networks are available and which network policies have been applied to their applications. In the future we could also create network diagrams to allow for inspection during a production problem or a security audit.

During this thesis we did not consider the creation of virtual networks from a central place as this was hard to achieve without a system that could distribute files to all agents. If such a system is in place we could make use of it to distribute the \gls{cni} plugins and their configuration files for the virtual networks, allowing developers to create new virtual networks using the custom controller that we implemented. Once the creation of virtual networks is implemented we can also provide the list of available virtual networks back to the users for them to select a network for their container.

Another extension could be a plugin model for cloud providers. At the moment operators need to manually configure load balancers as ingress points on a cloud platform to point to the right services, also called North-South traffic for a cluster. In the future we could create a plugin per cloud provider that automatically configures the load balancers, certificates and \gls{dns} records to point to the right services. We could also think of chaining different kind of plugins together. For example a plugin that logs all packets and another plugin that does traffic shaping of the same traffic.

We could also use the newly available information from the api-server to determine bottlenecks within clusters. When we have the list of current networks and policies we can leverage existing tools to monitor the network. Together with machine learning we could optimise traffic flows and apply the new network configuration using the same api-server. This could be a form of self adapting networks where we can ask the operator for confirmation or apply the new configuration automatically.
