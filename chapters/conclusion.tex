\chapter{Conclusion}
\label{chap:conclusions}
%\emph{This chapter contains the analysis and interpretation of the results. The research questions are answered as best as possible with the results that were obtained. The analysis also discussed parts of the questions that were left unanswered. An important topic is the validity of the results. What methods of validation were used? Could the results be generalised to other cases? What threats to validity can be identified? There is room here to discuss the results of related scientific literature here as well. How do the results obtained here relate to other work, and what consequences are there? Did your approach work better or worse? Did you learn anything new compared to the already existing body of knowledge? Finally, what could you say in hindsight on the research approach by followed? What could have done better? What lessons have been learned? What could other researchers use from your experience? A separate section should be devoted to future work", i.e., possible extension points of your work that you have identified. Even other researchers should be able to use those as a starting point.}
This thesis describes the difficulties in managing virtual networks on \gls{dcos}. For operators it is possible to create virtual networks using \gls{cni} plugins and some have support for network policies. However there is no central place in \gls{dcos} to see and configure those networks and policies. Unlike in Kubernetes where operators can manage the virtual networks by providing a desired state, the cluster then takes care of configuring the virtual networks and network policies. We first explored the different technologies in use and explained how they all work. This gives us a good foundation to be able to understand the difficulties in managing virtual container networks. Next we conducted a research on the features that are missing to manage those networks and how other vendors such as Kubernetes handle this.

Next a \gls{poc} was built to improve the management of virtual networks on \gls{dcos}. This \gls{poc} demonstrates that it is possible to abstract different \gls{sdn} solutions into one central \gls{api}. This \gls{api} can be used by operators to create new policies regardless of the plugins implemented within the cluster. Furthermore it enables developers to debug and see which policies are blocking traffic to their applications. Giving insight to developers deemed crucial as developers would end up removing their applications from the virtual network to make them work. This is an unwanted situation at Lunatech's automotive client because they need to isolate different parts of their data processing pipeline for privacy and security concerns. The newly built api-server is able to provide useful insights into the state of the network.

To conclude we can say that the new solution improves the management of virtual networks on \gls{dcos}. As operators and developers are able to retrieve and configure the virtual network state from a central place instead of having to dig into the concrete implementations of each \gls{sdn} provider.
