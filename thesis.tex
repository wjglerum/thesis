\documentclass[svgnames]{uvamscse} %Add xcolor parameter here to prevent conflict.

\usepackage[utf8]{inputenc}
\usepackage[english]{babel}
\usepackage{booktabs}
\usepackage{array}
\usepackage{paralist}
\usepackage{longtable}
\usepackage{amsmath,amssymb,amsthm}
\usepackage{csquotes}
\usepackage{xcolor}
\usepackage{framed}
\usepackage{listings}
\usepackage{protobuf/lang}  % include language definition for protobuf
\usepackage{protobuf/style} % include custom style for proto declarations.
\usepackage[backend=biber,style=numeric]{biblatex}
\usepackage[colorinlistoftodos]{todonotes}
\usepackage{pgfgantt}
\usepackage{hyperref}
\usepackage[toc,acronym,nogroupskip,nohypertypes={acronym}]{glossaries}
\usepackage{subcaption}
\hypersetup{pdfstartview={XYZ null null null}}
\colorlet{shadecolor}{yellow}

%Title page
\title{Improving management of virtual container networks on DC/OS}
\author{Willem Jan Glerum - 11317493}
\authemail{wjglerum@gmail.com}
\host{Lunatech Labs B.V. \url{https://lunatech.com} }
\supervisor{dr. P. Grosso \url{p.grosso@uva.nl}}
\hostsupervisor{dr. A. Haxaire \url{adrien.haxaire@lunatech.nl}}
\date{\today}
%Random virtual network image
\coverpic[200pt]{images/virtual.png}

\abstract{
Developers use containers to deploy their applications highly available on container orchestration platforms. On Mesosphere DC/OS it is possible to isolate services with virtual container networks and network policies provided by CNI plugins. Isolation is often needed for privacy and security concerns and for separating different tenants on the platform. At the time of writing, there is no method in place to centrally manage virtual networks and network policies on DC/OS.

The goal of this thesis is to understand the current problems in managing virtual container networks on DC/OS and to propose solutions to these problems. To do this, the study first gives background information on the used technologies to be able to identify the current problems. Based on this study, a proof of concept is presented. It is able to centrally manage network policies for different vendors.

The solution proposed in this thesis shows that it is possible to improve the management of virtual networks without a radical change in the architecture of DC/OS. Operators are now able to retrieve the current state of applied network policies and adjust them as needed.
}

%This section summarises the content of the thesis for potential readers who do not have time to read it whole,  or for those undecided whether to read it at all. Sum up the following aspects:
%    \item relevance and motivation for the research
%    \item research questions and a brief description of the research method
%    \item results, contributions and conclusions

%Styles for the lstlisting blocks.
%\input{lststyles}

%Glossary register:
\newacronym{cni}{CNI}{Container Network Interface}
\newacronym{sdn}{SDN}{Software Defined Networking}
\newacronym{poc}{PoC}{Proof of Concept}
\newacronym{ucr}{UCR}{Universal Container Runtime}
\newacronym{iot}{IoT}{Internet of Things}
\newacronym{dcos}{DC/OS}{Datacenter Operating System}

\makeglossaries

%Bibliography DB
\addbibresource{refs.bib}

%Useful definitions for (long)tables:
\newcolumntype{L}[1]{>{\raggedright\let\newline\\\arraybackslash\hspace{0pt}}p{#1}}
\newcolumntype{C}[1]{>{\centering\let\newline\\\arraybackslash\hspace{0pt}}p{#1}}
\newcolumntype{R}[1]{>{\raggedleft\let\newline\\\arraybackslash\hspace{0pt}}p{#1}}

\begin{document}

\maketitle
\chapter{Introduction}
Test\cite{calico}

\chapter{Background and context}
\label{chap:background}
%\emph{This chapter contains all the information needed to put the thesis into context. It is common to use a revised version of your literature survey for this purpose. It is important to explicitly refer from your text to sources you have used, they will be listed in your bibliography.}

In this chapter we give more background information about the technology stack relevant to this thesis. Not all technologies are in use at the Lunatech's automotive client at the moment, nevertheless we include them to provide more information about similar technologies or new technologies that we could use in the future. We start with containers and container orchestration platforms which form the basics of running containerised workloads. Next we explore the different virtual networks technologies with \gls{cni}.

\section{Containers}
\label{sec:containers}
A Linux container is an isolated set of resources on a compute host. Containers work with namespacing: a container can only access resources in the same namespace and not any other namespaces. Meaning that a container can only see and modify its own processes and not other processes running in another container or on the host itself. Except for privileged containers which can use any resource and have access to every process on the host. 

Furthermore every container will get its own isolated network stack preventing privileged access to sockets and interfaces on other containers. One can create links and bridges to allow IP connectivity to and from a container.

Resource utilisation for containers is limited with \glspl{cgroup}. Giving each container a restricted set of resources, preventing the container to bring down the host by using too much RAM or CPU\cite{docker_security}.

\subsection{Docker}
\label{subsec:docker}
Docker is the standard in industry for running Linux containers. The Docker Engine is a container platform that makes it easy to run your container. You can run a container on your local developer machine and ship it, to run in a production environment.

It works with a Dockerfile to describe how to build your image. Every step will create a new file system layer and all layers together are your Docker image. You can reuse these layers between containers so that you can make a generic application container and add a layer on top for the specific application. This allows for reusability and will save storage and bandwidth when sharing images. The most popular way to share images is the Docker Hub, however one can also host a private Docker Registry to store and serve your own images.

\subsection{Kubernetes}
\label{subsec:kubernetes}
Kubernetes is an open-source container orchestration platform build from scratch, based on the ideas from the internal container engine used at Google. The basic scheduling unit in Kubernetes is a pod. A pod is collection of one or more containers that are co-located together on a host and share an unique IP address preventing port conflicts in the cluster. Pods allow you to run tightly coupled applications together.

Kubernetes consists of different loosely coupled components as can be seen in Figure~\ref{fig:k8s-arch}. The masters run a key-value store called etcd\cite{etcd}, the API server and a scheduler component. An operator gives the cluster the desired state by running commands to the API server and stores its state in etcd. The scheduler makes sure that the desired state is achieved by for example choosing and instructing the nodes to run or delete the pods.

\begin{figure}
    \centering
    \includegraphics[width=1\columnwidth]{images/k8s-arch}
    \caption{Kubernetes architecture diagram\cite{k8s_arch}}
    \label{fig:k8s-arch}
\end{figure}

The Kubernetes nodes, also called workers or minions, run the actual workloads which are Linux containers using the Docker Engine. The kubelet component manages the state of the node which will accept resource offerings from the scheduler. The kube-proxy is responsible for managing the network on the node. It routes incoming traffic to the pod addresses and it can expose pods bundled as a service to the cluster, serving as a load balancer to the pods or services. 

\subsection{DC/OS}
\label{subsec:dcos}
\Gls{dcos} describes itself in the documentation\cite{dcos_what}: 
\begin{displayquote}
``As a datacenter operating system, DC/OS is itself a distributed system, a cluster manager, a container platform, and an operating system.''
\end{displayquote} 
Unlike a traditional operating system \gls{dcos} runs as a distributed operating system on different nodes and each node has its own host operating system. The master nodes accept tasks and schedule them on the agent nodes. Apache Mesos\cite{apache_mesos}, a distributed systems kernel, is responsible for the lifecycle of tasks. It launches schedulers on the slaves based on the type of tasks. Figure~\ref{fig:dcos-arch} shows an overview of all the components in a \gls{dcos} cluster. \Gls{dcos} has two built-in schedulers for scheduling Linux containers and two different container runtimes which can be seen in the container orchestration box on the master nodes in Figure~\ref{fig:dcos-arch}. 

\begin{figure}
    \centering
    \includegraphics[width=1\columnwidth]{images/dcos-arch}
    \caption{\gls{dcos} architecture diagram\cite{dcos_arch}}
    \label{fig:dcos-arch}
\end{figure}

Marathon\cite{marathon} runs the actual container orchestration engine, allowing to schedule tasks and frameworks. A task could be an application inside a Docker image and a framework could be Kafka\cite{kafka} or Cassandra\cite{cassandra} which manage their own life cycle using Mesos. These frameworks and other services can be installed as a package from the Mesosphere Universe\cite{universe}. To periodically schedule jobs, operators can use Metronome\cite{metronome}, which allows to run jobs once or based on a regular schedule. On \gls{dcos} operators are not only limited to running containers, as binary executable can be launched and containerised at runtime. The Docker Engine is present on every slave to run the Docker images or operators can choose to use the Mesos \gls{ucr}\cite{ucr} to run Docker images and binary executables.

The technologies and components mentioned above make it possible to run different kind of workloads, not just containerised workloads. We can even run Kubernetes as a framework on \gls{dcos}, this way the Kubernetes scheduler will ask Mesos for resource offerings.

\section{CNI}
\label{sec:cni}
To deal with networking and containers the \gls{cni} specification was created, describing how to interact with network interfaces in Linux containers. This allows providers to write plugins to configure \gls{sdn} with container workloads. Mesos, \gls{dcos}, Kubernetes and even \gls{aws} have support for \gls{cni} plugins. Allowing operators to create virtual networks on container orchestration platforms.
The \gls{cni} specification only deals with creating and deleting network resources, it does not do network policies for example. Network policies need to be managed by the plugin itself, as every plugin has a different goal and ways of enforcing policies. A similar project exists for the Docker daemon called \gls{cnm}\cite{cnm, dua2016learning} implemented as libnetwork where developers can also develop new plugins. For example Project Calico provides plugins for both \gls{cni} and \gls{cnm} and both are installed when using the universe package from \gls{dcos}.

There is a set of standard provided plugins\cite{cni_plugin} that allow for basic operations like creating network interfaces and \gls{ipam}. Different kinds of interfaces can be created such as bridges, loopback interfaces and \glspl{vlan}. IP addresses can be either requested from a \gls{dhcp} server or maintained by a local database of allocated IPs. Other plugin providers can provide their own implementations of assigning IP address to containers.

\subsection{Project Calico}
\label{subsec:calico}
Project Calico uses a pure Layer 3 approach to enable virtual networking between container workloads. As opposed to traditional solutions which in most cases use overlay networks. Overlay networks have the extra burden of en- and decapsulating packages between workloads when the traffic crosses different machines. Without this Calico is able to save CPU cycles and makes it easier to understand packets on the wire. Performance tests\cite{dzone, dataplane, chunqi} show that Calico is able to achieve nearly identical throughput to directly connected workloads. Overlay networks, and other plugins such as Weave\cite{weave} are not able to reach this throughput.

Instead of using a vSwitch, Calico uses a vRouter to route traffic between containers. This vRouter uses the Layer 3 forwarding technique found in the Linux kernel. A local agent, called Felix, programs the routes from the containers to the host in the node's route table. Furthermore, a \gls{bird} is running on every node to advertise routes to containers on other nodes in the cluster. Figure~\ref{fig:hops} shows the hops made between two different containers or workloads. State information is exchanged using \gls{bgp} or route reflectors.

By default all containers can talk to each other in a Calico network. Operators want to separate traffic between different tenants of the compute cluster, by assigning policies. A policy is translated to a rule in iptables\cite{iptables} on the host, leveraging the default Linux firewall. This can even be extended to specific rules for different workloads within a tenancy. The policies are shared using a distributed key-value store called etcd\cite{etcd}, as can be seen in Figure~\ref{fig:calico-arch}.

\begin{figure}
    \centering
    \includegraphics[width=1\columnwidth]{images/calico-hops}
    \caption{IP connectivity in Calico\cite{calico_learn}}
    \label{fig:hops}
\end{figure}

\subsection{Cilium}
\label{subsec:cilium}
A plugin that also takes a whole different approach is Cilium\cite{cilium} which allows for \gls{api} aware security policies. Instead of only operating at the network level with host and port based firewall rules, Cilium allows filtering of different protocols like \gls{http}, \gls{grpc}\cite{grpc} and Kafka\cite{kafka}. They do this by leveraging a new Linux kernel technology called \gls{bpf}\cite{mccanne1993bsd, cilium_bpf} by dynamically inserting \gls{bpf} bytecode into the Linux kernel, as can be seen in Figure~\ref{fig:cilium-arch}. 

This \gls{cni} plugin creates \gls{bpf} programs in the Linux kernel that control the network access. The \gls{bpf} programs are \gls{jit} compiled to CPU instructions to allow for native execution performance. Operators can do a lot of different things with this technology: simply monitoring the packets, redirecting packets and blocking packets. Only recent versions of the Linux kernel have \gls{bpf} capabilities, version 4.8.0 or newer is required. 

\begin{figure}
    \centering
    \begin{subfigure}[b]{0.49\textwidth}
    \includegraphics[width=0.8\textwidth]{images/calico-arch}
    \caption{Project Calico component overview\cite{calico_about}}
    \label{fig:calico-arch}
    \end{subfigure}
    \begin{subfigure}[b]{0.49\textwidth}
    \includegraphics[width=\textwidth]{images/cilium-arch}
    \caption{Cilium component overview\cite{cilium_concepts}}
    \label{fig:cilium-arch}
    \end{subfigure}
\end{figure}

\chapter{Problems with Virtual Container Networks}
\label{chap:research}
%\emph{This chapter reports on the execution of the research method as described in an earlier chapter. If the research has been divided into phases, they are introduced, reported on and concluded individually. If needed, this chapter could be split up to balance out the sizes of all chapters.}
 First we list the features are required by operators to manage virtual networks in Section~\ref{sec:required-features}, secondly we explain the current difficulties with managing virtual networks on \gls{dcos} in Section~\ref{sec:current-state}. Next we present the approach taken by another popular container orchestration platform, Kubernetes, to handle virtual networks in Section~\ref{sec:k8s-virtual-networks}. To finish we provide an overview of the problems at the end of this chapter in Section~\ref{sec:problems}.

\section{Required features}
\label{sec:required-features}
 We created a list of required features that an operator should be able to execute to manage virtual networks and policies on a container orchestration platform. Below is the list of the features with an accompanying explanation from the viewpoint of an operator of the cluster. As an operator I want to:
\begin{itemize}
    \item[\textit{list virtual networks:}] to see which networks are available in my cluster and see properties such as the name, tags, subnets and available IP addresses.
    \item[\textit{select virtual network:}] to select a virtual network to launch my workload on.
    \item[\textit{create virtual network:}] to provision and create new virtual networks in my cluster without manually configuring every agent in the cluster.
    \item[\textit{list network policies:}] to get an overview of all the configured network policies in the cluster, which can be filtered based on service, tenant or label to search for specific rules that are blocking traffic.
    \item[\textit{create network policy:}] to apply new firewall policies in the cluster without worrying about the different implementations of different plugin vendors.
    \item[\textit{know the available IPv4 addresses:}] to determine if the solution is usable for the number of services in the cluster.
    \item[\textit{have IPv6 support:}] to allow for native IPV6 support for my services and a bigger pool of available IP address for the growing number of applications with their own IP address.
\end{itemize}

\section{Current state of virtual networks in DC/OS}
\label{sec:current-state}
\Gls{dcos} provides virtual networks by itself as overlay networks which can be used by the Docker and Mesos containerizer. This overlay is prepacked and enabled by default on the cluster. \Gls{cni} plugins are also supported on \gls{dcos} and can be used by both containerizers. There are three different networking modes available for containers on \gls{dcos}:
\begin{itemize}
    \item[\textbf{Host networking}] The container shares the same network namespace as the host, this restricts the ports an application can use as the TCP/UDP port range is shared with the host.
    \item[\textbf{Bridge networking}] The container is launched on a bridge interface, has its own network namespace and applications are only reachable over port mappings to the host.
    \item[\textbf{Container networking}]  The container has its own network namespace and the underlying network is provided by \gls{cni} plugins. This networking mode is the most flexible as it allows operators to run applications on any port without dealing with port conflicts or port mappings.
\end{itemize}

\subsection{DC/OS virtual networks}
\label{subsec:dcos-virtual-networks}
The overlay network was introduced to enable an IP-per-container model in \gls{dcos}. This allows operators to run applications on the default ports without worrying about port conflicts. This is done by creating an overlay network using \gls{vxlan}\cite{mahalingam2014virtual} which is supported by the Linux kernel. It works by creating two network bridges on each host, one for each containerizer. Containers on the same host and bridge can communicate directly with each other over the network bridge. A packet from a Mesos container to a Docker container on the same host will be routed through both bridges. A packet from a Mesos container on Agent~1 to a Docker container on Agent~2 follows a different path. First the packet will be routed to the Mesos bridge, the host's network stack consumes the package and encapsulates using \gls{vxlan} on Agent~1. Next Agent~2 decapsulates this packet and sends it up to the Docker bridge to be sent to the Docker container as can be seen in Figure~\ref{fig:dcos-overlay-arch}. A container gets assigned a virtual network interface with the help of the \gls{dcos} overlay \gls{cni} plugin.

\begin{figure}
    \centering
    \includegraphics[width=1\columnwidth]{images/dcos-overlay-arch}
    \caption{DC/OS overlay in action\cite{dcos_overlay_arch}}
    \label{fig:dcos-overlay-arch}
\end{figure}

IP addresses are managed by the agent itself instead of a central location. A central location would require a reliably and consistent way of IP address assignment to containers in the cluster. The default configuration uses a \texttt{9.0.0.0/8} subnet, this subnet is divided in smaller chunks: a \texttt{/24} to be managed by every agent itself. On the agent this subnet is divided into two equal subnets for each containerizer, resulting in 32 usable IP addresses for each container type on every agent. The agent's overlay module can request its allocated subnet from the overlay module running on the masters based on its identifier. To launch a container on a virtual network an operator can select the required virtual network from the \gls{dcos} web interface when configuring a task as \gls{dcos} is aware of the default overlay networks.

The default overlay network currently has the following limitations:
\begin{itemize}
    \item A limit on the number of usable IP addresses on each host, determined by the network prefix size. The default maximum is 32 Mesos and 32 Docker containers.
    \item Only IPv6 support for Docker containers, Mesos containers will fail to start as the overlay network does not support IPv6 for Mesos containers.
    \item Operators are only able to create and delete virtual networks during the installation of the cluster.
    \item The length of network names is limited to 15 characters by the Linux kernel.
    \item Marathon cannot execute health checks in the default configurations as Marathon itself is not running on the virtual network.
    \item No \gls{api} to create and manage network policies on the virtual networks.
\end{itemize}

\subsection{CNI plugins on DC/OS}
\label{subsec:dcos-cni}
To allow for more flexibility with virtual networks on \gls{dcos}, operators can also chose to create virtual networks with other providers. Plugins following the \gls{cni} specification as discussed in Section~\ref{sec:cni} can be installed on \gls{dcos}. This works by placing the plugin in a dedicated folder on every agent. Networks can be created and configured by placing their relevant configuration files on every agent. Plugin location: \texttt{/opt/mesosphere/active/cni/} and configuration location: \texttt{/opt/mesosphere/etc/dcos/network/cni/}. A restart of the Mesos agent is required to make use of the new plugin or new configuration. The method mentioned above is only possible when using the Mesos containerizer. For Docker we need to create a Docker network on every agents with the \gls{cni} plugin as its driver.

Next any container can be launched on a virtual network by adding the following JSON to the configuration of the application: \texttt{\{"ipAddress": \{"networkName": "<<name>>"\}\}}. This will instruct the Mesos scheduler to request an IP address from that network using the configured \gls{cni} plugin. The plugin will take care of \gls{ipam} and creating and deleting of the network interface on the host machine. Depending on the features of the \gls{cni} plugin, network policies can be applied by using the \gls{api} of the plugin.

The \gls{cni} plugin model for \gls{dcos} makes it flexible for operators to configure different virtual networks. However there is a drawback to this approach, as these networks are not visible to the operators and users of the cluster, as stated on the documentation\cite{dcos_sdn}:
\begin{displayquote}
    ``NOTE: The network tab currently displays information about containers that are associated with virtual networks managed by DC/OS overlay. It does not have information about containers running on virtual networks managed by any other CNI/CNM provider.'' 
\end{displayquote}
This makes it confusing for developers who want to deploy their applications. Colleagues on the client's project are having a hard time to understand how their applications are deployed and how to connect to them. Containers now live on a different network and are not always available on every machine. For example the SSH jumpbox used to connect to the cluster is not setup with Calico and is unaware of the virtual networks. It does not have the routes in its route table to connect to the container workloads on a virtual network. 

There is also a problem with the different types of \gls{dns} providers within \gls{dcos}. The mesos-dns component provides a \gls{fqdn} to every service with the following structure: \texttt{<service-name>.<group-name>.<framework-name>.mesos}. This domain name however returns the IP address of the host of the container. In the case that the container is running on a virtual network it will return the wrong IP address. A connection is not possible because the port is only in use on the container network interface, resulting in a time-out for a user request. It is possible with a non recommended workaround to instruct Mesos to return the container IP address instead of the host IP address\cite{mesos_workaround}. The other and newer \gls{dns} provider is dcos-dns providing different addresses based on the network type of the container, these entries are summarised in Table~\ref{tab:dcos-dns} together with the mesos-dns.

\begin{table}[ht]
\centering
\begin{tabular}{l l l} 
 \textbf{DNS name} & \textbf{Container IP} & \textbf{Host IP} \\
 \hline\hline
 \texttt{*.autoip.dcos.thisdcos.directory} & Container networking & Host or bridge networking \\ 
 \texttt{*.agentip.dcos.thisdcos.directory} & & \checkmark \\
 \texttt{*.containerip.dcos.thisdcos.directory} & \checkmark & \\
 \texttt{*.mesos} & \checkmark & \\
 \hline
\end{tabular}
\caption{Different DNS entries for \gls{dcos} tasks}
\label{tab:dcos-dns}
\end{table}

To summarise we have the following drawbacks of using \gls{cni} plugins on \gls{dcos} to create virtual networks:
\begin{itemize}
    \item No method to retrieve or select the virtual networks created with \gls{cni} in the \gls{dcos} web and command line interface. Operators can specify a virtual network name for a container to attach it to that network, however there is no place to see the available networks created with \gls{cni} plugins.
    \item No generic method of configuring network policies, as each plugin provider has its own dedicated \gls{api}.
    \item An operator manually has to distribute the plugin and configuration files to every slave using a custom build solution.
    \item Different \gls{dns} providers return different IP addresses for container workloads on a virtual network.
\end{itemize}

\section{Virtual networks in Kubernetes}
\label{sec:k8s-virtual-networks}
There are different ways of connecting containers in Kubernetes. The most basic one is in a pod where containers in the same pod can communicate via localhost. Pods get their own unique IP address, called the IP-per-pod model, preventing operators to deal with port mappings and \gls{nat} to connect to pods from another pod or host. This network can be implemented by a \gls{cni} plugin or other available\gls{sdn} solutions\cite{k8s-network} which are available in Kubernetes. Furthermore if a plugin supports network policies an operator can leverage the plugin to configure network policies in the cluster. All the virtual network related options can be configured using the kube-apiserver as discussed in Section~\ref{subsec:kubernetes}, such as configuring the network provider and managing its network policies regardless of the chosen implementation. This makes it easy and transparent to the operators of a Kubernetes cluster.

To connect to a pod, from another pod or externally through a load balancer, we cannot use the pod IP address as this IP can change when pods are being rescheduled by the scheduler or when we have a set of the same pods for high availability. We can define a service which uses the kube-proxy, which is present on every node, to route the requests to the right set of pods. By default traffic from the client is captured by iptables\cite{iptables} on the host and redirected to be forwarded by the kube-proxy.

\section{Problems with virtual networks on DC/OS}
\label{sec:problems}
In the first section of this chapter we showed that operators can choose to launch a container on a virtual network provided by \gls{dcos}. Using the default overlay network, operators can select the network from the web interface as \gls{dcos} is aware of the networks that have been created with the overlay module. However this is not the case with virtual networks provided by \gls{cni} plugins. Table~\ref{tab:problems} gives an overview of the virtual network related features available in \gls{dcos} and Kubernetes. As you can see Kubernetes does provide most of the features already as it stores the network state with the kube-apiserver.

\begin{table}
\centering
\renewcommand*{\arraystretch}{1.4}
\begin{tabular}{|p{4cm}|p{2.5cm}|p{4cm}|p{2.5cm}|}\hline 
 \textbf{Feature} & \textbf{\gls{dcos} overlay networks} & \textbf{\gls{dcos} \gls{cni} plugins} & \textbf{Kubernetes}\\ \hline
 \textit{List virtual networks} & Available & Operator has to manual inspect the \gls{cni} configuration files on every host & Available \\ \hline
 \textit{Select virtual network} & Available & Operator has to manual provide a virtual network name based on the list obtained from above & Available \\ \hline
 \textit{Create virtual network} & Only at installation time & Manual on every host & Available \\ \hline
 \textit{List network policies} & Network policies are not provided & Operator has to use the plugin specific \gls{api}  & Available \\ \hline
 \textit{Create network policy} & Network policies are not provided & Operator has to use the plugin specific \gls{api} & Available \\ \hline
 \textit{Available IPv4 addresses} & 32 per containerizer per host & Plugin specific & Plugin specific \\ \hline
 \textit{IPv6 support} & Only for Docker containers & Plugin specific & Plugin specific \\ \hline
\end{tabular}
\caption{Available features for virtual container networks}
\label{tab:problems}
\end{table}

\chapter{Proof of Concept}
\label{chap:proof-of-concept}
%\emph{This chapter presents and clarifies the results obtained during the research. The focus should be on the factual results, not the interpretation or discussion. Tables and graphics should be used to increase the clarity of the results where applicable.}
\section{Expanding CNI}
\label{sec:expanding-cni}

\section{Networkobject API}
\label{sec:networkobject-api}

\chapter{Validation of PoC}
\label{chap:validation}
\todo[inline]{Report on how the PoC answers RQ2 and addresses the issues RQ1}
This chapter reports on how the \gls{poc} from Chapter~\ref{chap:proof-of-concept} answers \textbf{RQ2} and how it addresses the issues from \textbf{RQ1} as described in Chapter~\ref{chap:research}.

\chapter{Conclusions}
\label{chap:conclusions}
%\emph{This chapter contains the analysis and interpretation of the results. The research questions are answered as best as possible with the results that were obtained. The analysis also discussed parts of the questions that were left unanswered. An important topic is the validity of the results. What methods of validation were used? Could the results be generalised to other cases? What threats to validity can be identified? There is room here to discuss the results of related scientific literature here as well. How do the results obtained here relate to other work, and what consequences are there? Did your approach work better or worse? Did you learn anything new compared to the already existing body of knowledge? Finally, what could you say in hindsight on the research approach by followed? What could have done better? What lessons have been learned? What could other researchers use from your experience? A separate section should be devoted to future work", i.e., possible extension points of your work that you have identified. Even other researchers should be able to use those as a starting point.}

\todo[inline]{Write conclusions, around halve a page}
\section{Future work}
\lable{sec:future-work}
\chapter{Future work}
\label{chap:future-work}
At the moment the \gls{poc} includes an entire Kubernetes cluster, ofcourse this is overkill for a final solution. As future work we could implement our own api-server for the network objects and make it run highly available on all the masters in a \gls{dcos} cluster. Furthermore we need to extend the dcos-cli to include the new network objects that we have defined. Once this is in place we can extend the web interface too, the web interface could query the api-server for network objects and display them in the web interface. This will help developers and operators to check which virtual networks are available and which network policies have been applied to their applications. In the future we could also create network diagrams to allow for inspection during a production problem or a security audit.

During this thesis we did not consider the creation of virtual networks from a central place as this was hard to achieve without a system that could distribute files to all agents. If such a system is in place we could make use of it to distribute the \gls{cni} plugins and their configuration files for the virtual networks. Allowing developers to create new virtual networks using the custom controller that we implemented. Once the creation of virtual networks works we can also provide the list of available virtual networks back to the users for them to select a network for their container.

Another extension could be a plugin model for cloud providers. At the moment operators need to manually configure load balancers as ingress points on a cloud platform to point to the right services, also called North-South traffic for a cluster. In the future we could create a plugin per cloud provider that automatically configures the load balancers, certificates and \gls{dns} records to point to the right services. We could also think of chaining different kind of plugins together. For example a plugin that logs all packets and another plugin that does traffic shaping of the same traffic.

We could also use the newly available information from the api-server to determine bottlenecks within clusters. When we have the list of current networks and policies we can leverage existing tools to monitor the network. Together with for example machine learning we could optimise traffic flows and apply the new network configuration using the same api-server. This could be a form of self adapting networks where we can ask the operator for confirmation or apply the new configuration automatically.


\clearpage

\addcontentsline{toc}{chapter}{Bibliography}
\printbibliography{}
\printglossaries

\appendix
\glsresetall 

\end{document}
